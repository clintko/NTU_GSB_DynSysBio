
% Default to the notebook output style

    


% Inherit from the specified cell style.




    
\documentclass{article}

    
    
    \usepackage{graphicx} % Used to insert images
    \usepackage{adjustbox} % Used to constrain images to a maximum size 
    \usepackage{color} % Allow colors to be defined
    \usepackage{enumerate} % Needed for markdown enumerations to work
    \usepackage{geometry} % Used to adjust the document margins
    \usepackage{amsmath} % Equations
    \usepackage{amssymb} % Equations
    \usepackage{eurosym} % defines \euro
    \usepackage[mathletters]{ucs} % Extended unicode (utf-8) support
    \usepackage[utf8x]{inputenc} % Allow utf-8 characters in the tex document
    \usepackage{fancyvrb} % verbatim replacement that allows latex
    \usepackage{grffile} % extends the file name processing of package graphics 
                         % to support a larger range 
    % The hyperref package gives us a pdf with properly built
    % internal navigation ('pdf bookmarks' for the table of contents,
    % internal cross-reference links, web links for URLs, etc.)
    \usepackage{hyperref}
    \usepackage{longtable} % longtable support required by pandoc >1.10
    \usepackage{booktabs}  % table support for pandoc > 1.12.2
    

    
    
    \definecolor{orange}{cmyk}{0,0.4,0.8,0.2}
    \definecolor{darkorange}{rgb}{.71,0.21,0.01}
    \definecolor{darkgreen}{rgb}{.12,.54,.11}
    \definecolor{myteal}{rgb}{.26, .44, .56}
    \definecolor{gray}{gray}{0.45}
    \definecolor{lightgray}{gray}{.95}
    \definecolor{mediumgray}{gray}{.8}
    \definecolor{inputbackground}{rgb}{.95, .95, .85}
    \definecolor{outputbackground}{rgb}{.95, .95, .95}
    \definecolor{traceback}{rgb}{1, .95, .95}
    % ansi colors
    \definecolor{red}{rgb}{.6,0,0}
    \definecolor{green}{rgb}{0,.65,0}
    \definecolor{brown}{rgb}{0.6,0.6,0}
    \definecolor{blue}{rgb}{0,.145,.698}
    \definecolor{purple}{rgb}{.698,.145,.698}
    \definecolor{cyan}{rgb}{0,.698,.698}
    \definecolor{lightgray}{gray}{0.5}
    
    % bright ansi colors
    \definecolor{darkgray}{gray}{0.25}
    \definecolor{lightred}{rgb}{1.0,0.39,0.28}
    \definecolor{lightgreen}{rgb}{0.48,0.99,0.0}
    \definecolor{lightblue}{rgb}{0.53,0.81,0.92}
    \definecolor{lightpurple}{rgb}{0.87,0.63,0.87}
    \definecolor{lightcyan}{rgb}{0.5,1.0,0.83}
    
    % commands and environments needed by pandoc snippets
    % extracted from the output of `pandoc -s`
    \DefineVerbatimEnvironment{Highlighting}{Verbatim}{commandchars=\\\{\}}
    % Add ',fontsize=\small' for more characters per line
    \newenvironment{Shaded}{}{}
    \newcommand{\KeywordTok}[1]{\textcolor[rgb]{0.00,0.44,0.13}{\textbf{{#1}}}}
    \newcommand{\DataTypeTok}[1]{\textcolor[rgb]{0.56,0.13,0.00}{{#1}}}
    \newcommand{\DecValTok}[1]{\textcolor[rgb]{0.25,0.63,0.44}{{#1}}}
    \newcommand{\BaseNTok}[1]{\textcolor[rgb]{0.25,0.63,0.44}{{#1}}}
    \newcommand{\FloatTok}[1]{\textcolor[rgb]{0.25,0.63,0.44}{{#1}}}
    \newcommand{\CharTok}[1]{\textcolor[rgb]{0.25,0.44,0.63}{{#1}}}
    \newcommand{\StringTok}[1]{\textcolor[rgb]{0.25,0.44,0.63}{{#1}}}
    \newcommand{\CommentTok}[1]{\textcolor[rgb]{0.38,0.63,0.69}{\textit{{#1}}}}
    \newcommand{\OtherTok}[1]{\textcolor[rgb]{0.00,0.44,0.13}{{#1}}}
    \newcommand{\AlertTok}[1]{\textcolor[rgb]{1.00,0.00,0.00}{\textbf{{#1}}}}
    \newcommand{\FunctionTok}[1]{\textcolor[rgb]{0.02,0.16,0.49}{{#1}}}
    \newcommand{\RegionMarkerTok}[1]{{#1}}
    \newcommand{\ErrorTok}[1]{\textcolor[rgb]{1.00,0.00,0.00}{\textbf{{#1}}}}
    \newcommand{\NormalTok}[1]{{#1}}
    
    % Define a nice break command that doesn't care if a line doesn't already
    % exist.
    \def\br{\hspace*{\fill} \\* }
    % Math Jax compatability definitions
    \def\gt{>}
    \def\lt{<}
    % Document parameters
    \title{???\_PS11\_Langevin}
    
    
    

    % Pygments definitions
    
\makeatletter
\def\PY@reset{\let\PY@it=\relax \let\PY@bf=\relax%
    \let\PY@ul=\relax \let\PY@tc=\relax%
    \let\PY@bc=\relax \let\PY@ff=\relax}
\def\PY@tok#1{\csname PY@tok@#1\endcsname}
\def\PY@toks#1+{\ifx\relax#1\empty\else%
    \PY@tok{#1}\expandafter\PY@toks\fi}
\def\PY@do#1{\PY@bc{\PY@tc{\PY@ul{%
    \PY@it{\PY@bf{\PY@ff{#1}}}}}}}
\def\PY#1#2{\PY@reset\PY@toks#1+\relax+\PY@do{#2}}

\expandafter\def\csname PY@tok@cs\endcsname{\let\PY@it=\textit\def\PY@tc##1{\textcolor[rgb]{0.25,0.50,0.50}{##1}}}
\expandafter\def\csname PY@tok@kc\endcsname{\let\PY@bf=\textbf\def\PY@tc##1{\textcolor[rgb]{0.00,0.50,0.00}{##1}}}
\expandafter\def\csname PY@tok@s2\endcsname{\def\PY@tc##1{\textcolor[rgb]{0.73,0.13,0.13}{##1}}}
\expandafter\def\csname PY@tok@k\endcsname{\let\PY@bf=\textbf\def\PY@tc##1{\textcolor[rgb]{0.00,0.50,0.00}{##1}}}
\expandafter\def\csname PY@tok@gu\endcsname{\let\PY@bf=\textbf\def\PY@tc##1{\textcolor[rgb]{0.50,0.00,0.50}{##1}}}
\expandafter\def\csname PY@tok@gr\endcsname{\def\PY@tc##1{\textcolor[rgb]{1.00,0.00,0.00}{##1}}}
\expandafter\def\csname PY@tok@go\endcsname{\def\PY@tc##1{\textcolor[rgb]{0.53,0.53,0.53}{##1}}}
\expandafter\def\csname PY@tok@mh\endcsname{\def\PY@tc##1{\textcolor[rgb]{0.40,0.40,0.40}{##1}}}
\expandafter\def\csname PY@tok@sx\endcsname{\def\PY@tc##1{\textcolor[rgb]{0.00,0.50,0.00}{##1}}}
\expandafter\def\csname PY@tok@bp\endcsname{\def\PY@tc##1{\textcolor[rgb]{0.00,0.50,0.00}{##1}}}
\expandafter\def\csname PY@tok@c1\endcsname{\let\PY@it=\textit\def\PY@tc##1{\textcolor[rgb]{0.25,0.50,0.50}{##1}}}
\expandafter\def\csname PY@tok@w\endcsname{\def\PY@tc##1{\textcolor[rgb]{0.73,0.73,0.73}{##1}}}
\expandafter\def\csname PY@tok@ow\endcsname{\let\PY@bf=\textbf\def\PY@tc##1{\textcolor[rgb]{0.67,0.13,1.00}{##1}}}
\expandafter\def\csname PY@tok@cp\endcsname{\def\PY@tc##1{\textcolor[rgb]{0.74,0.48,0.00}{##1}}}
\expandafter\def\csname PY@tok@nd\endcsname{\def\PY@tc##1{\textcolor[rgb]{0.67,0.13,1.00}{##1}}}
\expandafter\def\csname PY@tok@gh\endcsname{\let\PY@bf=\textbf\def\PY@tc##1{\textcolor[rgb]{0.00,0.00,0.50}{##1}}}
\expandafter\def\csname PY@tok@gp\endcsname{\let\PY@bf=\textbf\def\PY@tc##1{\textcolor[rgb]{0.00,0.00,0.50}{##1}}}
\expandafter\def\csname PY@tok@nc\endcsname{\let\PY@bf=\textbf\def\PY@tc##1{\textcolor[rgb]{0.00,0.00,1.00}{##1}}}
\expandafter\def\csname PY@tok@nn\endcsname{\let\PY@bf=\textbf\def\PY@tc##1{\textcolor[rgb]{0.00,0.00,1.00}{##1}}}
\expandafter\def\csname PY@tok@nl\endcsname{\def\PY@tc##1{\textcolor[rgb]{0.63,0.63,0.00}{##1}}}
\expandafter\def\csname PY@tok@no\endcsname{\def\PY@tc##1{\textcolor[rgb]{0.53,0.00,0.00}{##1}}}
\expandafter\def\csname PY@tok@m\endcsname{\def\PY@tc##1{\textcolor[rgb]{0.40,0.40,0.40}{##1}}}
\expandafter\def\csname PY@tok@kt\endcsname{\def\PY@tc##1{\textcolor[rgb]{0.69,0.00,0.25}{##1}}}
\expandafter\def\csname PY@tok@nf\endcsname{\def\PY@tc##1{\textcolor[rgb]{0.00,0.00,1.00}{##1}}}
\expandafter\def\csname PY@tok@gt\endcsname{\def\PY@tc##1{\textcolor[rgb]{0.00,0.27,0.87}{##1}}}
\expandafter\def\csname PY@tok@sb\endcsname{\def\PY@tc##1{\textcolor[rgb]{0.73,0.13,0.13}{##1}}}
\expandafter\def\csname PY@tok@nb\endcsname{\def\PY@tc##1{\textcolor[rgb]{0.00,0.50,0.00}{##1}}}
\expandafter\def\csname PY@tok@sc\endcsname{\def\PY@tc##1{\textcolor[rgb]{0.73,0.13,0.13}{##1}}}
\expandafter\def\csname PY@tok@mb\endcsname{\def\PY@tc##1{\textcolor[rgb]{0.40,0.40,0.40}{##1}}}
\expandafter\def\csname PY@tok@vi\endcsname{\def\PY@tc##1{\textcolor[rgb]{0.10,0.09,0.49}{##1}}}
\expandafter\def\csname PY@tok@mo\endcsname{\def\PY@tc##1{\textcolor[rgb]{0.40,0.40,0.40}{##1}}}
\expandafter\def\csname PY@tok@kn\endcsname{\let\PY@bf=\textbf\def\PY@tc##1{\textcolor[rgb]{0.00,0.50,0.00}{##1}}}
\expandafter\def\csname PY@tok@gd\endcsname{\def\PY@tc##1{\textcolor[rgb]{0.63,0.00,0.00}{##1}}}
\expandafter\def\csname PY@tok@il\endcsname{\def\PY@tc##1{\textcolor[rgb]{0.40,0.40,0.40}{##1}}}
\expandafter\def\csname PY@tok@gi\endcsname{\def\PY@tc##1{\textcolor[rgb]{0.00,0.63,0.00}{##1}}}
\expandafter\def\csname PY@tok@se\endcsname{\let\PY@bf=\textbf\def\PY@tc##1{\textcolor[rgb]{0.73,0.40,0.13}{##1}}}
\expandafter\def\csname PY@tok@sh\endcsname{\def\PY@tc##1{\textcolor[rgb]{0.73,0.13,0.13}{##1}}}
\expandafter\def\csname PY@tok@na\endcsname{\def\PY@tc##1{\textcolor[rgb]{0.49,0.56,0.16}{##1}}}
\expandafter\def\csname PY@tok@c\endcsname{\let\PY@it=\textit\def\PY@tc##1{\textcolor[rgb]{0.25,0.50,0.50}{##1}}}
\expandafter\def\csname PY@tok@kp\endcsname{\def\PY@tc##1{\textcolor[rgb]{0.00,0.50,0.00}{##1}}}
\expandafter\def\csname PY@tok@sd\endcsname{\let\PY@it=\textit\def\PY@tc##1{\textcolor[rgb]{0.73,0.13,0.13}{##1}}}
\expandafter\def\csname PY@tok@nt\endcsname{\let\PY@bf=\textbf\def\PY@tc##1{\textcolor[rgb]{0.00,0.50,0.00}{##1}}}
\expandafter\def\csname PY@tok@cm\endcsname{\let\PY@it=\textit\def\PY@tc##1{\textcolor[rgb]{0.25,0.50,0.50}{##1}}}
\expandafter\def\csname PY@tok@vg\endcsname{\def\PY@tc##1{\textcolor[rgb]{0.10,0.09,0.49}{##1}}}
\expandafter\def\csname PY@tok@ni\endcsname{\let\PY@bf=\textbf\def\PY@tc##1{\textcolor[rgb]{0.60,0.60,0.60}{##1}}}
\expandafter\def\csname PY@tok@s1\endcsname{\def\PY@tc##1{\textcolor[rgb]{0.73,0.13,0.13}{##1}}}
\expandafter\def\csname PY@tok@nv\endcsname{\def\PY@tc##1{\textcolor[rgb]{0.10,0.09,0.49}{##1}}}
\expandafter\def\csname PY@tok@mf\endcsname{\def\PY@tc##1{\textcolor[rgb]{0.40,0.40,0.40}{##1}}}
\expandafter\def\csname PY@tok@vc\endcsname{\def\PY@tc##1{\textcolor[rgb]{0.10,0.09,0.49}{##1}}}
\expandafter\def\csname PY@tok@sr\endcsname{\def\PY@tc##1{\textcolor[rgb]{0.73,0.40,0.53}{##1}}}
\expandafter\def\csname PY@tok@kr\endcsname{\let\PY@bf=\textbf\def\PY@tc##1{\textcolor[rgb]{0.00,0.50,0.00}{##1}}}
\expandafter\def\csname PY@tok@si\endcsname{\let\PY@bf=\textbf\def\PY@tc##1{\textcolor[rgb]{0.73,0.40,0.53}{##1}}}
\expandafter\def\csname PY@tok@o\endcsname{\def\PY@tc##1{\textcolor[rgb]{0.40,0.40,0.40}{##1}}}
\expandafter\def\csname PY@tok@kd\endcsname{\let\PY@bf=\textbf\def\PY@tc##1{\textcolor[rgb]{0.00,0.50,0.00}{##1}}}
\expandafter\def\csname PY@tok@mi\endcsname{\def\PY@tc##1{\textcolor[rgb]{0.40,0.40,0.40}{##1}}}
\expandafter\def\csname PY@tok@err\endcsname{\def\PY@bc##1{\setlength{\fboxsep}{0pt}\fcolorbox[rgb]{1.00,0.00,0.00}{1,1,1}{\strut ##1}}}
\expandafter\def\csname PY@tok@ge\endcsname{\let\PY@it=\textit}
\expandafter\def\csname PY@tok@ne\endcsname{\let\PY@bf=\textbf\def\PY@tc##1{\textcolor[rgb]{0.82,0.25,0.23}{##1}}}
\expandafter\def\csname PY@tok@s\endcsname{\def\PY@tc##1{\textcolor[rgb]{0.73,0.13,0.13}{##1}}}
\expandafter\def\csname PY@tok@gs\endcsname{\let\PY@bf=\textbf}
\expandafter\def\csname PY@tok@ss\endcsname{\def\PY@tc##1{\textcolor[rgb]{0.10,0.09,0.49}{##1}}}

\def\PYZbs{\char`\\}
\def\PYZus{\char`\_}
\def\PYZob{\char`\{}
\def\PYZcb{\char`\}}
\def\PYZca{\char`\^}
\def\PYZam{\char`\&}
\def\PYZlt{\char`\<}
\def\PYZgt{\char`\>}
\def\PYZsh{\char`\#}
\def\PYZpc{\char`\%}
\def\PYZdl{\char`\$}
\def\PYZhy{\char`\-}
\def\PYZsq{\char`\'}
\def\PYZdq{\char`\"}
\def\PYZti{\char`\~}
% for compatibility with earlier versions
\def\PYZat{@}
\def\PYZlb{[}
\def\PYZrb{]}
\makeatother


    % Exact colors from NB
    \definecolor{incolor}{rgb}{0.0, 0.0, 0.5}
    \definecolor{outcolor}{rgb}{0.545, 0.0, 0.0}



    
    % Prevent overflowing lines due to hard-to-break entities
    \sloppy 
    % Setup hyperref package
    \hypersetup{
      breaklinks=true,  % so long urls are correctly broken across lines
      colorlinks=true,
      urlcolor=blue,
      linkcolor=darkorange,
      citecolor=darkgreen,
      }
    % Slightly bigger margins than the latex defaults
    
    \geometry{verbose,tmargin=1in,bmargin=1in,lmargin=1in,rmargin=1in}
    
    

    \begin{document}
    
    
    \maketitle
    
    

    
    \begin{Verbatim}[commandchars=\\\{\}]
{\color{incolor}In [{\color{incolor}1}]:} \PY{k+kn}{import} \PY{n+nn}{numpy} \PY{k}{as} \PY{n+nn}{np}
        \PY{k+kn}{import} \PY{n+nn}{matplotlib}\PY{n+nn}{.}\PY{n+nn}{pyplot} \PY{k}{as} \PY{n+nn}{plt}
        \PY{o}{\PYZpc{}}\PY{k}{matplotlib} inline
\end{Verbatim}

    \section{Brownian Motion (Standard Wiener
Process)}\label{brownian-motion-standard-wiener-process}

    The formal definition of Brownian Motion is \[dx = 0*dt + dW \]

where W(t) is a standard Wiener process, such that

\[ W(t) - W(s) = \sqrt{t-s} * N(0, 1)\]

where N(0, 1) denotes a normally distributed random variable with zero
men and unit variance.

By applying Euler-Maruyama method, the Brownian Motion could be viewed
as an iterated sum of independent identical normal random variables.

    First, I simulated several Brownian Motion with specified number of time
points. Then, the simulated processes are ploted with different colors.

    \begin{Verbatim}[commandchars=\\\{\}]
{\color{incolor}In [{\color{incolor}2}]:} \PY{c}{\PYZsh{} set parameters}
        \PY{n}{T} \PY{o}{=} \PY{l+m+mi}{1}                     \PY{c}{\PYZsh{}\PYZsh{} the total time simulated}
        \PY{n}{N} \PY{o}{=} \PY{l+m+mi}{500}                   \PY{c}{\PYZsh{}\PYZsh{} the number of time points within time = 0 \PYZti{} T}
        \PY{n}{dt} \PY{o}{=} \PY{n}{T}\PY{o}{/}\PY{n}{N}                  \PY{c}{\PYZsh{}\PYZsh{} the length of dt}
        \PY{n}{t} \PY{o}{=} \PY{n}{np}\PY{o}{.}\PY{n}{linspace}\PY{p}{(}\PY{l+m+mi}{0}\PY{p}{,} \PY{n}{T}\PY{p}{,} \PY{n}{N}\PY{p}{)}  \PY{c}{\PYZsh{}\PYZsh{} time points}
        \PY{n}{sampleNum} \PY{o}{=} \PY{l+m+mi}{5}            \PY{c}{\PYZsh{}\PYZsh{} number of brownian process simulated}
        
        \PY{c}{\PYZsh{} create iid standard normal random variables multiplied by squared root of dt}
        \PY{n}{dW} \PY{o}{=} \PY{n}{dt}\PY{o}{*}\PY{o}{*}\PY{l+m+mf}{0.5} \PY{o}{*} \PY{n}{np}\PY{o}{.}\PY{n}{random}\PY{o}{.}\PY{n}{multivariate\PYZus{}normal}\PY{p}{(}\PY{p}{[}\PY{l+m+mi}{0}\PY{p}{]} \PY{o}{*} \PY{n}{N}\PY{p}{,} \PY{n}{np}\PY{o}{.}\PY{n}{identity}\PY{p}{(}\PY{n}{N}\PY{p}{)}\PY{p}{,} \PY{n}{sampleNum}\PY{p}{)}
        
        \PY{c}{\PYZsh{} simulate brownian process}
        \PY{n}{W} \PY{o}{=} \PY{n}{np}\PY{o}{.}\PY{n}{array}\PY{p}{(}\PY{p}{[} \PY{p}{[}\PY{n+nb}{sum}\PY{p}{(}\PY{n}{process}\PY{p}{[}\PY{l+m+mi}{0}\PY{p}{:}\PY{n}{idx}\PY{p}{]}\PY{p}{)} \PY{k}{for} \PY{n}{idx} \PY{o+ow}{in} \PY{n+nb}{range}\PY{p}{(}\PY{n}{N}\PY{p}{)}\PY{p}{]} \PY{k}{for} \PY{n}{process} \PY{o+ow}{in} \PY{n}{dW} \PY{p}{]}\PY{p}{)}
        
        \PY{c}{\PYZsh{} Plot the Brownian Process}
        \PY{n}{fig}\PY{p}{,} \PY{n}{ax} \PY{o}{=} \PY{n}{plt}\PY{o}{.}\PY{n}{subplots}\PY{p}{(}\PY{l+m+mi}{1}\PY{p}{,} \PY{l+m+mi}{1}\PY{p}{,} \PY{n}{figsize}\PY{o}{=}\PY{p}{(}\PY{l+m+mi}{10}\PY{p}{,}\PY{l+m+mi}{3}\PY{p}{)}\PY{p}{)}
        \PY{n}{ax}\PY{o}{.}\PY{n}{plot}\PY{p}{(}\PY{n}{t}\PY{p}{,} \PY{n}{W}\PY{o}{.}\PY{n}{T}\PY{p}{)}
        \PY{n}{ax}\PY{o}{.}\PY{n}{set\PYZus{}title}\PY{p}{(}\PY{l+s}{\PYZdq{}}\PY{l+s}{Brownian Process}\PY{l+s}{\PYZdq{}}\PY{p}{)}
\end{Verbatim}

            \begin{Verbatim}[commandchars=\\\{\}]
{\color{outcolor}Out[{\color{outcolor}2}]:} <matplotlib.text.Text at 0x727dba8>
\end{Verbatim}
        
    \begin{center}
    \adjustimage{max size={0.9\linewidth}{0.9\paperheight}}{柯逵悅_PS11_Langevin_files/柯逵悅_PS11_Langevin_4_1.png}
    \end{center}
    { \hspace*{\fill} \\}
    
    \section{Function of the process}\label{function-of-the-process}

    In the document writen by Prof.~Higham, he evaluated the function of a
stochastic process, which has the form as
follows:\\\[ u(W(t)) = exp(t + \frac{1}{2} * W(t)) \] where W(t) is a
standard Wiener process. The solution (solid thick line) could be
obtained by calculate the means of several simulated process (thin
dashed line)

    \begin{Verbatim}[commandchars=\\\{\}]
{\color{incolor}In [{\color{incolor}3}]:} \PY{c}{\PYZsh{} calculate the function of each simulated process}
        \PY{n}{uW} \PY{o}{=} \PY{n}{np}\PY{o}{.}\PY{n}{e}\PY{o}{*}\PY{o}{*}\PY{p}{(}\PY{n}{t} \PY{o}{+} \PY{l+m+mf}{0.5} \PY{o}{*} \PY{n}{W}\PY{p}{)}
        
        \PY{c}{\PYZsh{} mean of brownian Process}
        \PY{n}{umean} \PY{o}{=} \PY{n}{np}\PY{o}{.}\PY{n}{mean}\PY{p}{(}\PY{n}{uW}\PY{p}{,} \PY{n}{axis} \PY{o}{=} \PY{l+m+mi}{0}\PY{p}{)}
        
        \PY{c}{\PYZsh{} plot the process and mean}
        \PY{n}{fig}\PY{p}{,} \PY{n}{ax} \PY{o}{=} \PY{n}{plt}\PY{o}{.}\PY{n}{subplots}\PY{p}{(}\PY{l+m+mi}{1}\PY{p}{,} \PY{l+m+mi}{1}\PY{p}{,} \PY{n}{figsize} \PY{o}{=} \PY{p}{(}\PY{l+m+mi}{10}\PY{p}{,} \PY{l+m+mi}{2}\PY{p}{)}\PY{p}{)}
        \PY{n}{ax}\PY{o}{.}\PY{n}{plot}\PY{p}{(}\PY{n}{t}\PY{p}{,} \PY{n}{uW}\PY{o}{.}\PY{n}{T}\PY{p}{,}    \PY{n}{linestyle} \PY{o}{=} \PY{l+s}{\PYZsq{}}\PY{l+s}{\PYZhy{}\PYZhy{}}\PY{l+s}{\PYZsq{}}\PY{p}{)}
        \PY{n}{ax}\PY{o}{.}\PY{n}{plot}\PY{p}{(}\PY{n}{t}\PY{p}{,} \PY{n}{umean}\PY{o}{.}\PY{n}{T}\PY{p}{,} \PY{n}{linestyle} \PY{o}{=} \PY{l+s}{\PYZsq{}}\PY{l+s}{\PYZhy{}}\PY{l+s}{\PYZsq{}}\PY{p}{,} \PY{n}{linewidth} \PY{o}{=} \PY{l+m+mi}{4}\PY{p}{)}
        \PY{n}{ax}\PY{o}{.}\PY{n}{set\PYZus{}ylim}\PY{p}{(}\PY{l+m+mi}{0}\PY{p}{,} \PY{l+m+mi}{5}\PY{p}{)}
\end{Verbatim}

            \begin{Verbatim}[commandchars=\\\{\}]
{\color{outcolor}Out[{\color{outcolor}3}]:} (0, 5)
\end{Verbatim}
        
    \begin{center}
    \adjustimage{max size={0.9\linewidth}{0.9\paperheight}}{柯逵悅_PS11_Langevin_files/柯逵悅_PS11_Langevin_7_1.png}
    \end{center}
    { \hspace*{\fill} \\}
    
    \section{Stochastic Version of a Linear
ODE}\label{stochastic-version-of-a-linear-ode}

    The Euler-Maryama method is a quick, powerful ideas on simulated a ODE
with noise.\\The example ODE here has the form:
\[ dx = f(x)dt + g(x)dW \]

\[ f(x) = λx \]

\[ g(x) = µx \] The λ and µ are both constants.

    \begin{Verbatim}[commandchars=\\\{\}]
{\color{incolor}In [{\color{incolor}4}]:} \PY{c}{\PYZsh{} define functions and function that simulated stochastic version of ODE}
        \PY{k}{def} \PY{n+nf}{f}\PY{p}{(}\PY{n}{x}\PY{p}{,} \PY{n}{lambdaNum}\PY{p}{)}\PY{p}{:}
            \PY{k}{return} \PY{n}{lambdaNum} \PY{o}{*} \PY{n}{x}
        
        \PY{k}{def} \PY{n+nf}{g}\PY{p}{(}\PY{n}{x}\PY{p}{,} \PY{n}{mu}\PY{p}{)}\PY{p}{:}
            \PY{k}{return} \PY{n}{mu} \PY{o}{*} \PY{n}{x}
        
        \PY{k}{def} \PY{n+nf}{process}\PY{p}{(}\PY{n}{x\PYZus{}0} \PY{o}{=} \PY{l+m+mi}{0}\PY{p}{,} \PY{n}{iterations} \PY{o}{=} \PY{l+m+mi}{10}\PY{p}{,} \PY{n}{sampleNum} \PY{o}{=} \PY{l+m+mi}{1}\PY{p}{,} \PY{n}{dt} \PY{o}{=} \PY{l+m+mf}{0.01}\PY{p}{,} \PY{n}{lambdaNum} \PY{o}{=} \PY{l+m+mi}{0}\PY{p}{,} \PY{n}{mu} \PY{o}{=} \PY{l+m+mi}{0}\PY{p}{)}\PY{p}{:}
            \PY{c}{\PYZsh{} initialization}
            \PY{n}{X} \PY{o}{=} \PY{p}{[}\PY{p}{]}
            \PY{n}{N} \PY{o}{=} \PY{n}{iterations}
            
            \PY{c}{\PYZsh{} solve ODE}
            \PY{k}{for} \PY{n}{dummyNum} \PY{o+ow}{in} \PY{n+nb}{range}\PY{p}{(}\PY{n}{sampleNum}\PY{p}{)}\PY{p}{:}
                \PY{c}{\PYZsh{} initial value of each simulation}
                \PY{n}{x} \PY{o}{=} \PY{p}{[}\PY{n}{x\PYZus{}0}\PY{p}{]}
                
                \PY{c}{\PYZsh{} each simulation of stochastic ODE}
                \PY{k}{for} \PY{n}{idx} \PY{o+ow}{in} \PY{n+nb}{range}\PY{p}{(}\PY{n}{N}\PY{p}{)}\PY{p}{:}
                    \PY{n}{dw} \PY{o}{=} \PY{n}{dt}\PY{o}{*}\PY{o}{*}\PY{l+m+mf}{0.5} \PY{o}{*} \PY{n+nb}{float}\PY{p}{(}\PY{n}{np}\PY{o}{.}\PY{n}{random}\PY{o}{.}\PY{n}{standard\PYZus{}normal}\PY{p}{(}\PY{l+m+mi}{1}\PY{p}{)}\PY{p}{)}
                    \PY{n}{x}\PY{o}{.}\PY{n}{append}\PY{p}{(} \PY{n}{x}\PY{p}{[}\PY{n}{idx}\PY{p}{]} \PY{o}{+} \PY{n}{f}\PY{p}{(}\PY{n}{x}\PY{p}{[}\PY{n}{idx}\PY{p}{]}\PY{p}{,} \PY{n}{lambdaNum}\PY{p}{)} \PY{o}{*} \PY{n}{dt} \PY{o}{+} \PY{n}{g}\PY{p}{(}\PY{n}{x}\PY{p}{[}\PY{n}{idx}\PY{p}{]}\PY{p}{,} \PY{n}{mu}\PY{p}{)} \PY{o}{*} \PY{n}{dw}\PY{p}{)}
                
                \PY{c}{\PYZsh{} add each simulation to a list}
                \PY{n}{X}\PY{o}{.}\PY{n}{append}\PY{p}{(}\PY{n}{x}\PY{p}{)}
            
            \PY{k}{return} \PY{n}{np}\PY{o}{.}\PY{n}{array}\PY{p}{(}\PY{n}{X}\PY{p}{)}
\end{Verbatim}

    \begin{Verbatim}[commandchars=\\\{\}]
{\color{incolor}In [{\color{incolor}17}]:} \PY{c}{\PYZsh{} set parameters}
         \PY{n}{x\PYZus{}0} \PY{o}{=} \PY{l+m+mf}{0.01}        \PY{c}{\PYZsh{}\PYZsh{} initial value of x}
         \PY{n}{N} \PY{o}{=} \PY{l+m+mi}{500}           \PY{c}{\PYZsh{}\PYZsh{} number of simulated process}
         \PY{n}{dt} \PY{o}{=} \PY{l+m+mi}{1}
         \PY{n}{iterations} \PY{o}{=} \PY{l+m+mi}{100}
         
         \PY{c}{\PYZsh{} simulate the solution}
         \PY{c}{\PYZsh{}\PYZsh{} no noise (set mu = 0)}
         \PY{n}{N} \PY{o}{=} \PY{l+m+mi}{1}
         \PY{n}{iterations} \PY{o}{=} \PY{l+m+mi}{10}
         \PY{n}{result01} \PY{o}{=} \PY{n}{process}\PY{p}{(}\PY{n}{x\PYZus{}0}\PY{o}{=}\PY{n}{x\PYZus{}0}\PY{p}{,} \PY{n}{sampleNum}\PY{o}{=}\PY{n}{N}\PY{p}{,} \PY{n}{iterations}\PY{o}{=}\PY{n}{iterations}\PY{p}{,} \PY{n}{dt}\PY{o}{=}\PY{n}{dt}\PY{p}{,} \PY{n}{lambdaNum}\PY{o}{=}\PY{l+m+mi}{1}\PY{p}{,} \PY{n}{mu}\PY{o}{=}\PY{l+m+mi}{0}\PY{p}{)}
         
         \PY{c}{\PYZsh{}\PYZsh{} only noise (set lambda = 0)}
         \PY{n}{N} \PY{o}{=} \PY{l+m+mi}{100}
         \PY{n}{iterations} \PY{o}{=} \PY{l+m+mi}{10}
         \PY{n}{result02} \PY{o}{=} \PY{n}{process}\PY{p}{(}\PY{n}{x\PYZus{}0}\PY{o}{=}\PY{n}{x\PYZus{}0}\PY{p}{,} \PY{n}{sampleNum}\PY{o}{=}\PY{n}{N}\PY{p}{,} \PY{n}{iterations}\PY{o}{=}\PY{n}{iterations}\PY{p}{,} \PY{n}{dt}\PY{o}{=}\PY{n}{dt}\PY{p}{,} \PY{n}{lambdaNum}\PY{o}{=}\PY{l+m+mi}{0}\PY{p}{,} \PY{n}{mu}\PY{o}{=}\PY{l+m+mi}{1}\PY{p}{)}
         
         \PY{c}{\PYZsh{}\PYZsh{} only noise (set lambda = 0)}
         \PY{n}{N} \PY{o}{=} \PY{l+m+mi}{100}
         \PY{n}{iterations} \PY{o}{=} \PY{l+m+mi}{100}
         \PY{n}{result03} \PY{o}{=} \PY{n}{process}\PY{p}{(}\PY{n}{x\PYZus{}0}\PY{o}{=}\PY{n}{x\PYZus{}0}\PY{p}{,} \PY{n}{sampleNum}\PY{o}{=}\PY{n}{N}\PY{p}{,} \PY{n}{iterations}\PY{o}{=}\PY{n}{iterations}\PY{p}{,} \PY{n}{dt}\PY{o}{=}\PY{n}{dt}\PY{p}{,} \PY{n}{lambdaNum}\PY{o}{=}\PY{l+m+mi}{0}\PY{p}{,} \PY{n}{mu}\PY{o}{=}\PY{l+m+mi}{1}\PY{p}{)}
         
         \PY{c}{\PYZsh{}\PYZsh{} Lambda != 0 and mu != 0}
         \PY{n}{N} \PY{o}{=} \PY{l+m+mi}{100}
         \PY{n}{iterations} \PY{o}{=} \PY{l+m+mi}{10}
         \PY{n}{result04} \PY{o}{=} \PY{n}{process}\PY{p}{(}\PY{n}{x\PYZus{}0}\PY{o}{=}\PY{n}{x\PYZus{}0}\PY{p}{,} \PY{n}{sampleNum}\PY{o}{=}\PY{n}{N}\PY{p}{,} \PY{n}{iterations}\PY{o}{=}\PY{n}{iterations}\PY{p}{,} \PY{n}{dt}\PY{o}{=}\PY{n}{dt}\PY{p}{,} \PY{n}{lambdaNum}\PY{o}{=}\PY{l+m+mi}{10}\PY{p}{,} \PY{n}{mu}\PY{o}{=}\PY{l+m+mi}{10}\PY{p}{)}
         
         \PY{c}{\PYZsh{} plot the result}
         \PY{n}{fig}\PY{p}{,} \PY{n}{ax} \PY{o}{=} \PY{n}{plt}\PY{o}{.}\PY{n}{subplots}\PY{p}{(}\PY{l+m+mi}{4}\PY{p}{,} \PY{l+m+mi}{1}\PY{p}{,} \PY{n}{figsize} \PY{o}{=} \PY{p}{(}\PY{l+m+mi}{5}\PY{p}{,} \PY{l+m+mi}{10}\PY{p}{)}\PY{p}{)}
         \PY{n}{plt}\PY{o}{.}\PY{n}{subplots\PYZus{}adjust}\PY{p}{(}\PY{n}{hspace} \PY{o}{=} \PY{l+m+mf}{0.7}\PY{p}{)}
         \PY{n}{ax}\PY{p}{[}\PY{l+m+mi}{0}\PY{p}{]}\PY{o}{.}\PY{n}{plot}\PY{p}{(}\PY{n}{result01}\PY{o}{.}\PY{n}{T}\PY{p}{)}
         \PY{n}{ax}\PY{p}{[}\PY{l+m+mi}{0}\PY{p}{]}\PY{o}{.}\PY{n}{set\PYZus{}title}\PY{p}{(}\PY{l+s}{\PYZdq{}}\PY{l+s}{Lambda = 1, mu = 0 }\PY{l+s+se}{\PYZbs{}n}\PY{l+s}{ =\PYZgt{} Exponential Growth}\PY{l+s}{\PYZdq{}}\PY{p}{)}
         \PY{n}{ax}\PY{p}{[}\PY{l+m+mi}{1}\PY{p}{]}\PY{o}{.}\PY{n}{plot}\PY{p}{(}\PY{n}{result02}\PY{o}{.}\PY{n}{T}\PY{p}{)}
         \PY{n}{ax}\PY{p}{[}\PY{l+m+mi}{1}\PY{p}{]}\PY{o}{.}\PY{n}{set\PYZus{}title}\PY{p}{(}\PY{l+s}{\PYZdq{}}\PY{l+s}{Lambda = 0, mu = 1 }\PY{l+s+se}{\PYZbs{}n}\PY{l+s}{ Number of Process = 100 }\PY{l+s+se}{\PYZbs{}n}\PY{l+s}{ Number of Time Point = 10}\PY{l+s}{\PYZdq{}}\PY{p}{)}
         \PY{n}{ax}\PY{p}{[}\PY{l+m+mi}{2}\PY{p}{]}\PY{o}{.}\PY{n}{plot}\PY{p}{(}\PY{n}{result03}\PY{o}{.}\PY{n}{T}\PY{p}{)}
         \PY{n}{ax}\PY{p}{[}\PY{l+m+mi}{2}\PY{p}{]}\PY{o}{.}\PY{n}{set\PYZus{}title}\PY{p}{(}\PY{l+s}{\PYZdq{}}\PY{l+s}{Lambda = 0, mu = 1 }\PY{l+s+se}{\PYZbs{}n}\PY{l+s}{ Number of Process = 100 }\PY{l+s+se}{\PYZbs{}n}\PY{l+s}{ Number of Time Point = 100}\PY{l+s}{\PYZdq{}}\PY{p}{)}
         \PY{n}{ax}\PY{p}{[}\PY{l+m+mi}{3}\PY{p}{]}\PY{o}{.}\PY{n}{plot}\PY{p}{(}\PY{n}{result04}\PY{o}{.}\PY{n}{T}\PY{p}{)}
         \PY{n}{ax}\PY{p}{[}\PY{l+m+mi}{3}\PY{p}{]}\PY{o}{.}\PY{n}{set\PYZus{}title}\PY{p}{(}\PY{l+s}{\PYZdq{}}\PY{l+s}{Lambda = 10, mu = 10 }\PY{l+s+se}{\PYZbs{}n}\PY{l+s}{ Number of Process = 100 }\PY{l+s+se}{\PYZbs{}n}\PY{l+s}{ Number of Time Point = 10}\PY{l+s}{\PYZdq{}}\PY{p}{)}
\end{Verbatim}

            \begin{Verbatim}[commandchars=\\\{\}]
{\color{outcolor}Out[{\color{outcolor}17}]:} <matplotlib.text.Text at 0xfc47a58>
\end{Verbatim}
        
    \begin{center}
    \adjustimage{max size={0.9\linewidth}{0.9\paperheight}}{柯逵悅_PS11_Langevin_files/柯逵悅_PS11_Langevin_11_1.png}
    \end{center}
    { \hspace*{\fill} \\}
    
    In the second picture, where mu = 1 and lambda = 0, the noise in next
data point is proportion to the value of the last step. Therefore, the
process spread seems to spread in the short iterations. However, when
the time point increase, the mean of the process is zero, since the
kernel of the noise is a normal random variable with mean equals to
zero. Interestingly, there are some simulated processes rise and
decrease dramatically, but eventually converge back to zero.

    \section{Stochastic Version of a Gene Expression
ODE}\label{stochastic-version-of-a-gene-expression-ode}

    The ODE of the gene expression is \[ \frac{dx}{dt} = \beta - \alpha*x \]

By adding the noise, the Langevin equation of the gene expression
becomes
\[ dx = (\beta - \alpha*x) * dt + \sqrt{\beta*dt} * N(0, 1) + \sqrt{\alpha*x*dt} * N(0, 1)\]

    \begin{Verbatim}[commandchars=\\\{\}]
{\color{incolor}In [{\color{incolor}6}]:} \PY{k}{def} \PY{n+nf}{geneExp1}\PY{p}{(}\PY{n}{x\PYZus{}0}\PY{o}{=}\PY{l+m+mi}{0}\PY{p}{,} \PY{n}{beta}\PY{o}{=}\PY{l+m+mi}{0}\PY{p}{,} \PY{n}{alpha}\PY{o}{=}\PY{l+m+mi}{0}\PY{p}{,} \PY{n}{iterations}\PY{o}{=}\PY{l+m+mi}{10}\PY{p}{,} \PY{n}{sampleNum} \PY{o}{=} \PY{l+m+mi}{1}\PY{p}{,} \PY{n}{dt} \PY{o}{=} \PY{l+m+mf}{0.01}\PY{p}{)}\PY{p}{:}
            \PY{c}{\PYZsh{} initialization}
            \PY{n}{X} \PY{o}{=} \PY{p}{[}\PY{p}{]}
            \PY{n}{N} \PY{o}{=} \PY{n}{iterations}
            
            \PY{k}{for} \PY{n}{dummyNum} \PY{o+ow}{in} \PY{n+nb}{range}\PY{p}{(}\PY{n}{sampleNum}\PY{p}{)}\PY{p}{:}
                \PY{c}{\PYZsh{} initial value of each simulation}
                \PY{n}{x} \PY{o}{=} \PY{p}{[}\PY{n}{x\PYZus{}0}\PY{p}{]}
                
                \PY{c}{\PYZsh{} simulation}
                \PY{k}{for} \PY{n}{idx} \PY{o+ow}{in} \PY{n+nb}{range}\PY{p}{(}\PY{n}{N}\PY{p}{)}\PY{p}{:}
                    \PY{n}{dx} \PY{o}{=} \PY{p}{(}\PY{n}{beta} \PY{o}{\PYZhy{}} \PY{n}{alpha} \PY{o}{*} \PY{n}{x}\PY{p}{[}\PY{n}{idx}\PY{p}{]}\PY{p}{)} \PY{o}{*} \PY{n}{dt}
                    \PY{n}{x}\PY{o}{.}\PY{n}{append}\PY{p}{(} \PY{n}{x}\PY{p}{[}\PY{n}{idx}\PY{p}{]} \PY{o}{+} \PY{n}{dx} \PY{p}{)}
                    
                \PY{c}{\PYZsh{} add each simulation to a list}
                \PY{n}{X}\PY{o}{.}\PY{n}{append}\PY{p}{(}\PY{n}{x}\PY{p}{)}
            
            \PY{k}{return} \PY{n}{np}\PY{o}{.}\PY{n}{array}\PY{p}{(}\PY{n}{X}\PY{p}{)}    
                
        \PY{k}{def} \PY{n+nf}{geneExp2}\PY{p}{(}\PY{n}{x\PYZus{}0}\PY{o}{=}\PY{l+m+mi}{0}\PY{p}{,} \PY{n}{beta}\PY{o}{=}\PY{l+m+mi}{0}\PY{p}{,} \PY{n}{alpha}\PY{o}{=}\PY{l+m+mi}{0}\PY{p}{,} \PY{n}{iterations}\PY{o}{=}\PY{l+m+mi}{10}\PY{p}{,} \PY{n}{sampleNum} \PY{o}{=} \PY{l+m+mi}{1}\PY{p}{,} \PY{n}{dt} \PY{o}{=} \PY{l+m+mf}{0.01}\PY{p}{)}\PY{p}{:}
            \PY{c}{\PYZsh{} initialization}
            \PY{n}{X} \PY{o}{=} \PY{p}{[}\PY{p}{]}
            \PY{n}{N} \PY{o}{=} \PY{n}{iterations}
            
            \PY{k}{for} \PY{n}{dummyNum} \PY{o+ow}{in} \PY{n+nb}{range}\PY{p}{(}\PY{n}{sampleNum}\PY{p}{)}\PY{p}{:}
                \PY{c}{\PYZsh{} initial value of each simulation}
                \PY{n}{x} \PY{o}{=} \PY{p}{[}\PY{n}{x\PYZus{}0}\PY{p}{]}
                
                \PY{c}{\PYZsh{} simulation}
                \PY{k}{for} \PY{n}{idx} \PY{o+ow}{in} \PY{n+nb}{range}\PY{p}{(}\PY{n}{N}\PY{p}{)}\PY{p}{:}
                    \PY{n}{dx} \PY{o}{=} \PY{p}{(}\PY{n}{beta} \PY{o}{\PYZhy{}} \PY{n}{alpha} \PY{o}{*} \PY{n}{x}\PY{p}{[}\PY{n}{idx}\PY{p}{]}\PY{p}{)} \PY{o}{*} \PY{n}{dt} 
                    \PY{n}{dx} \PY{o}{+}\PY{o}{=} \PY{p}{(}\PY{n}{beta} \PY{o}{*} \PY{n}{dt}\PY{p}{)}\PY{o}{*}\PY{o}{*}\PY{l+m+mf}{0.5} \PY{o}{*} \PY{n+nb}{float}\PY{p}{(}\PY{n}{np}\PY{o}{.}\PY{n}{random}\PY{o}{.}\PY{n}{standard\PYZus{}normal}\PY{p}{(}\PY{l+m+mi}{1}\PY{p}{)}\PY{p}{)}
                    \PY{n}{dx} \PY{o}{+}\PY{o}{=} \PY{p}{(}\PY{n}{alpha} \PY{o}{*} \PY{n}{x}\PY{p}{[}\PY{n}{idx}\PY{p}{]} \PY{o}{*} \PY{n}{dt}\PY{p}{)}\PY{o}{*}\PY{o}{*}\PY{l+m+mf}{0.5} \PY{o}{*} \PY{n+nb}{float}\PY{p}{(}\PY{n}{np}\PY{o}{.}\PY{n}{random}\PY{o}{.}\PY{n}{standard\PYZus{}normal}\PY{p}{(}\PY{l+m+mi}{1}\PY{p}{)}\PY{p}{)}
                    \PY{n}{x}\PY{o}{.}\PY{n}{append}\PY{p}{(} \PY{n}{x}\PY{p}{[}\PY{n}{idx}\PY{p}{]} \PY{o}{+} \PY{n}{dx} \PY{p}{)}
                    
                \PY{c}{\PYZsh{} add each simulation to a list}
                \PY{n}{X}\PY{o}{.}\PY{n}{append}\PY{p}{(}\PY{n}{x}\PY{p}{)}
            
            \PY{k}{return} \PY{n}{np}\PY{o}{.}\PY{n}{array}\PY{p}{(}\PY{n}{X}\PY{p}{)}
\end{Verbatim}

    By setting the initial value smaller than the ratio BETA/ALPHA, we could
observe how the gene expression level rise to the steady state. On the
other hand, when the inital value is set larger than the ratio
BETA/ALPHA, we could observe how the protein level decay to the steady
state.

    \begin{Verbatim}[commandchars=\\\{\}]
{\color{incolor}In [{\color{incolor}7}]:} \PY{c}{\PYZsh{} set parameters}
        \PY{n}{BETA} \PY{o}{=} \PY{l+m+mi}{100}
        \PY{n}{ALPHA} \PY{o}{=} \PY{l+m+mi}{5}
        \PY{n}{dt} \PY{o}{=} \PY{l+m+mf}{0.01}
        \PY{n}{iterations} \PY{o}{=} \PY{l+m+mi}{500}
        \PY{n}{sampleNum} \PY{o}{=} \PY{l+m+mi}{1}
        
        \PY{c}{\PYZsh{} simulation gene expression}
        \PY{n}{x\PYZus{}0} \PY{o}{=} \PY{l+m+mi}{2}
        \PY{n}{resGeneExp01} \PY{o}{=} \PY{n}{geneExp1}\PY{p}{(}\PY{n}{x\PYZus{}0}\PY{o}{=}\PY{n}{x\PYZus{}0}\PY{p}{,} \PY{n}{beta}\PY{o}{=}\PY{n}{BETA}\PY{p}{,} \PY{n}{alpha}\PY{o}{=}\PY{n}{ALPHA}\PY{p}{,} \PY{n}{iterations}\PY{o}{=}\PY{n}{iterations}\PY{p}{,} \PY{n}{sampleNum}\PY{o}{=}\PY{l+m+mi}{1}\PY{p}{,} \PY{n}{dt}\PY{o}{=}\PY{n}{dt}\PY{p}{)}
        \PY{n}{resGeneExp02} \PY{o}{=} \PY{n}{geneExp2}\PY{p}{(}\PY{n}{x\PYZus{}0}\PY{o}{=}\PY{n}{x\PYZus{}0}\PY{p}{,} \PY{n}{beta}\PY{o}{=}\PY{n}{BETA}\PY{p}{,} \PY{n}{alpha}\PY{o}{=}\PY{n}{ALPHA}\PY{p}{,} \PY{n}{iterations}\PY{o}{=}\PY{n}{iterations}\PY{p}{,} \PY{n}{sampleNum}\PY{o}{=}\PY{n}{sampleNum}\PY{p}{,} \PY{n}{dt}\PY{o}{=}\PY{n}{dt}\PY{p}{)}
        
        \PY{n}{x\PYZus{}0} \PY{o}{=} \PY{l+m+mi}{100}
        \PY{n}{resGeneExp11} \PY{o}{=} \PY{n}{geneExp1}\PY{p}{(}\PY{n}{x\PYZus{}0}\PY{o}{=}\PY{n}{x\PYZus{}0}\PY{p}{,} \PY{n}{beta}\PY{o}{=}\PY{n}{BETA}\PY{p}{,} \PY{n}{alpha}\PY{o}{=}\PY{n}{ALPHA}\PY{p}{,} \PY{n}{iterations}\PY{o}{=}\PY{n}{iterations}\PY{p}{,} \PY{n}{sampleNum}\PY{o}{=}\PY{l+m+mi}{1}\PY{p}{,} \PY{n}{dt}\PY{o}{=}\PY{n}{dt}\PY{p}{)}
        \PY{n}{resGeneExp12} \PY{o}{=} \PY{n}{geneExp2}\PY{p}{(}\PY{n}{x\PYZus{}0}\PY{o}{=}\PY{n}{x\PYZus{}0}\PY{p}{,} \PY{n}{beta}\PY{o}{=}\PY{n}{BETA}\PY{p}{,} \PY{n}{alpha}\PY{o}{=}\PY{n}{ALPHA}\PY{p}{,} \PY{n}{iterations}\PY{o}{=}\PY{n}{iterations}\PY{p}{,} \PY{n}{sampleNum}\PY{o}{=}\PY{n}{sampleNum}\PY{p}{,} \PY{n}{dt}\PY{o}{=}\PY{n}{dt}\PY{p}{)}
        
        \PY{c}{\PYZsh{} plot the simulation}
        \PY{n}{fig}\PY{p}{,} \PY{n}{ax} \PY{o}{=} \PY{n}{plt}\PY{o}{.}\PY{n}{subplots}\PY{p}{(}\PY{l+m+mi}{2}\PY{p}{,} \PY{l+m+mi}{1}\PY{p}{,} \PY{n}{figsize}\PY{o}{=}\PY{p}{(}\PY{l+m+mi}{5}\PY{p}{,} \PY{l+m+mi}{7}\PY{p}{)}\PY{p}{)}
        \PY{n}{ax}\PY{p}{[}\PY{l+m+mi}{0}\PY{p}{]}\PY{o}{.}\PY{n}{plot}\PY{p}{(}\PY{n}{resGeneExp01}\PY{o}{.}\PY{n}{T}\PY{p}{)}
        \PY{n}{ax}\PY{p}{[}\PY{l+m+mi}{0}\PY{p}{]}\PY{o}{.}\PY{n}{plot}\PY{p}{(}\PY{n}{resGeneExp02}\PY{o}{.}\PY{n}{T}\PY{p}{)}
        \PY{n}{ax}\PY{p}{[}\PY{l+m+mi}{0}\PY{p}{]}\PY{o}{.}\PY{n}{set\PYZus{}title}\PY{p}{(}\PY{l+s}{\PYZdq{}}\PY{l+s}{BETA/ALPHA = 20, Initial Value = 2}\PY{l+s}{\PYZdq{}}\PY{p}{)}
        \PY{n}{ax}\PY{p}{[}\PY{l+m+mi}{1}\PY{p}{]}\PY{o}{.}\PY{n}{plot}\PY{p}{(}\PY{n}{resGeneExp11}\PY{o}{.}\PY{n}{T}\PY{p}{)}
        \PY{n}{ax}\PY{p}{[}\PY{l+m+mi}{1}\PY{p}{]}\PY{o}{.}\PY{n}{plot}\PY{p}{(}\PY{n}{resGeneExp12}\PY{o}{.}\PY{n}{T}\PY{p}{)}
        \PY{n}{ax}\PY{p}{[}\PY{l+m+mi}{1}\PY{p}{]}\PY{o}{.}\PY{n}{set\PYZus{}title}\PY{p}{(}\PY{l+s}{\PYZdq{}}\PY{l+s}{BETA/ALPHA = 20, Initial Value = 100}\PY{l+s}{\PYZdq{}}\PY{p}{)}
\end{Verbatim}

            \begin{Verbatim}[commandchars=\\\{\}]
{\color{outcolor}Out[{\color{outcolor}7}]:} <matplotlib.text.Text at 0x726a198>
\end{Verbatim}
        
    \begin{center}
    \adjustimage{max size={0.9\linewidth}{0.9\paperheight}}{柯逵悅_PS11_Langevin_files/柯逵悅_PS11_Langevin_17_1.png}
    \end{center}
    { \hspace*{\fill} \\}
    
    Note that in the plot above, the blue line represents the deterministic
ODE of gene expression, and the green line is the stochastic version of
the gene expression.


    % Add a bibliography block to the postdoc
    
    
    
    \end{document}
